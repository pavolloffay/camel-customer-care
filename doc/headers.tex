\documentclass[a4paper]{article}

\usepackage{tabu}
\usepackage[utf8]{inputenc}
\usepackage[english]{babel}
\usepackage[T1]{fontenc}
\usepackage[margin=3cm]{geometry}
\usepackage{graphicx}
\usepackage{acronym}
\usepackage{marvosym} % Special symbols (e.g. \Letter, \Telefon, \Email)
\usepackage{blindtext} % Very good package for sample text, language specific (use e.g. \usepackage[ngerman]{babel} before). Way better than the lipsum package (for examples see: http://ctan.mirrorcatalogs.com/macros/latex/contrib/blindtext/blindtext.pdf)
\usepackage{caption} % Better captioning (e.g. when clicking on Link to figure, the whole figure is viewable, not only the description)
\usepackage{listings} % Code listings
\usepackage{color} % Use of colors
\usepackage{eurosym}
\usepackage{float}
\usepackage{comment}
\usepackage{booktabs}
\usepackage[table,xcdraw]{xcolor}
\usepackage{pdflscape}

% Project specific stuff
\usepackage{fancyvrb}
\usepackage[nottoc]{tocbibind} % Get bibliography, LOF, LOT and LOL into TOC (Without numbers!). From http://tex.stackexchange.com/questions/71129/bibliography-in-table-of-contents
%\usepackage[nottoc,numbib]{tocbibind} % Bibliography with number  in TOC. See http://tex.stackexchange.com/questions/8458/making-the-bibliography-appear-in-the-table-of-contents
% tocbibind general usage: ftp://ftp.tex.ac.uk/tex-archive/macros/latex/contrib/tocbibind/tocbibind.pdf

\usepackage{newclude} % removes some restrictions from \include (e.g. pagebreak before and after \include). Details: http://ctan.mackichan.com/macros/latex/contrib/frankenstein/newclude.pdf

\definecolor{mygreen}{rgb}{0,0.6,0}
\definecolor{mygray}{rgb}{0.5,0.5,0.5}
\definecolor{mymauve}{rgb}{0.58,0,0.82}
\definecolor{mybeige}{RGB}{255,213,122}

\lstset{ %
    backgroundcolor=\color{mybeige},   % choose the background color; you must add \usepackage{color} or \usepackage{xcolor}
    basicstyle=\footnotesize,        % the size of the fonts that are used for the code
    breakatwhitespace=false,         % sets if automatic breaks should only happen at whitespace
    breaklines=true,                 % sets automatic line breaking
    captionpos=b,                    % sets the caption-position to bottom
    commentstyle=\color{mygreen},    % comment style
    deletekeywords={...},            % if you want to delete keywords from the given language
    escapeinside={\%*}{*)},          % if you want to add LaTeX within your code
    extendedchars=true,              % lets you use non-ASCII characters; for 8-bits encodings only, does not work with UTF-8
    frame=single,                    % adds a frame around the code
    keepspaces=true,                 % keeps spaces in text, useful for keeping indentation of code (possibly needs columns=flexible)
    keywordstyle=\color{black},      % keyword style
    language=Octave,                 % the language of the code
    morekeywords={*,...},            % if you want to add more keywords to the set
    numbers=none,                    % where to put the line-numbers; possible values are (none, left, right)
    numbersep=5pt,                   % how far the line-numbers are from the code
    numberstyle=\tiny\color{mygray}, % the style that is used for the line-numbers
    rulecolor=\color{black},         % if not set, the frame-color may be changed on line-breaks within not-black text (e.g. comments (green here))
    showspaces=false,                % show spaces everywhere adding particular underscores; it overrides 'showstringspaces'
    showstringspaces=false,          % underline spaces within strings only
    showtabs=false,                  % show tabs within strings adding particular underscores
    stepnumber=2,                    % the step between two line-numbers. If it's 1, each line will be numbered
    stringstyle=\color{mymauve},     % string literal style
    tabsize=2,                       % sets default tabsize to 2 spaces
    title=\lstname                   % show the filename of files included with \lstinputlisting; also try caption instead of title
}

% Variables
\newcommand{\titlepageFile}{./titlepage/titlepage-english}
\newcommand{\referencesFile}{./references}
\newcommand{\templateSection}{./sections/template-section}
\newcommand{\exOneSection}{./sections/ex1}
\newcommand{\exTwoSection}{./sections/ex2}
\newcommand{\exThreeSection}{./sections/ex3}
\newcommand{\exFourSection}{./sections/ex4}

% For efficient document generation. Details: http://www.weinelt.de/latex/includeonly.html
\includeonly{
    \titlepageFile,
    \templateSection/template,
    \exOneSection/ex1,
    \exTwoSection/ex2,
    \exThreeSection/ex3,
    \exFourSection/ex4}

% Packages that should be at the end of the preamble

\usepackage[
    backend=bibtex,
    style=authoryear,
    sortlocale=en_EN,
    natbib=true,
    url=true, 
    doi=true,
    eprint=false
]{biblatex}

\usepackage[autostyle]{csquotes}
\addbibresource{\referencesFile}

\usepackage[ % Clickable references
	pdfstartview={FitV},
	pdftitle={CSS SS2015 Assignment 1 -- Group 25},
	pdfauthor={},
	pdfsubject={Compuational Social Simulation},
	colorlinks=false
]{hyperref}